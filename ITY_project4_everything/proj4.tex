\documentclass[11pt]{article}
\usepackage[a4paper, left=1.4cm, top=2.3cm, text={18.3cm, 25.2cm}]{geometry}
\usepackage[T1]{fontenc}
\usepackage[utf8]{inputenc}
\usepackage[slovak]{babel}
\usepackage{hyperref}
\usepackage{lmodern}
\usepackage{url}
\DeclareUrlCommand\url{\urlstyle{tt}}


\begin{document}
% -------------------------------------------------------- %
\begin{titlepage}
\begin{center}
    \Huge
	\textsc{Vysoké učení technické v~Brně}\\
    \huge
	{\textsc{Fakulta informačních technologií }}\\
	
	\vspace{\stretch{0.382}}
	\LARGE
	Typografia a publikovanie\,--\,4. projekt\\
	\Huge
    Citácie\\
	
	\vspace{\stretch{0.618}}
\end{center}
{\Large 18. Apríl 2024 \hfill
Jaroslav Mervart}
\end{titlepage}
% -------------------------------------------------------- %
\section*{Úvod}
Od jaskynných malieb, ich zjednodušovania do formy viet a konečne aj znakov, až po ich reprezentáciu v digitálnej podobe. Rýchlosť vývoja a šírenia písma bola a je značne ovplyvnená aj nástrojmi určenými na jeho šírenie~\cite{Jiricek} Počítače sa stali prostriedkom dennej komunikácie, individualita rukopisu sa vytratila.~\cite{someARt}

\section{Problematika}
Pre mnohých predtým neskúsených ľudí otvoril prechod komunikácie do digitálneho sveta dvere k typografii.~\cite{IBMstuff} To ale spôsobilo aj úpadok v kvalite využitia typografie, kvôli nedostatočnej zručnosti užívateľov v danom obore.~\cite{DobraTypogWhere} Do toho ešte vstúpili inherentné problémy \uv{rôznojazyčnosti}, ako napríklad mäkčene a dĺžne, a ich preklad z \uv{kameničtiny}.~\cite{FontyvCs} Cieľom tohto textu je pozrieť sa aspoň na pár z aspektov typografie.

\section{Kerning}
Začnime podrezaním -- \texttt{kerning}. Je to stále relevantná téma, pričom jej hlavný cieľ je zaručenie čitateľnosti daného textu.~\cite{Macek2008}. Ako príklad by som rád uviedol snahu o modularizáciu čínskych znakov:
    \begin{quote}
        \uv{Na základe teórie modulov sú dôležité ťahy pre tvorbu znakov. Moduly čínštiny znakov sa skladajú z jedného alebo viacerých znakových komponentov, ktoré sa skladajú z jedného alebo viacerých prvkov (ťahov).}~\cite{ConferenceChina}
    \end{quote}
To dokazuje, že podrezanie vie byť chúlostivá záležitosť -- stovky kombinácií ťahov to zaručia.

\section{Ligatúry}
Ligatúry zväzujú, resp. spájajú znaky v texte. To sa ale nevylučuje s podrezaním. Práve naopak, príliš dlhé ligatúry by mohli spôsobiť že by výsledný text bol vizuálne neestetický a ťažko čitateľný. 
Takéto problémy riešime napríklad pri imitáciách písaného textu. Jedným z problémov je fakt, že písmena na sebe musia na seba naväzovať presne a bez výnimiek.~\cite{Slabikar}

\section*{Záver}
Komunikácia je veľkou súčasťou ľudstva a ľudia reagujú na veci podľa ich významu pre nich samotných.~\cite{thisIsNotFair}. Preto majú typografické rozhodnutia pri písaní schopnosť vytvoriť očakávania nielen o identite autora, ale aj o autorite a dôveryhodnosti vytvoreného textu.~\cite{AUTUniversity} 
% -------------------------------------------------------- %
\newpage
\bibliographystyle{czplain}
\bibliography{proj4}
% -------------------------------------------------------- %
\end{document}
Komprimovany soubor s odevzdanymi soubory pojmenujte svym loginem (nedávejte soubory do adresáře).
