\documentclass[twocolumn,11pt]{article}
\usepackage[a4paper, left=1.4cm, top=2.3cm, text={18.3cm, 25.2cm}]{geometry}
\usepackage[T1]{fontenc}
\usepackage[utf8]{inputenc}
\usepackage[czech]{babel}
\usepackage{hyperref}

\usepackage{amsthm}
\usepackage{amsmath}
\usepackage{array}
\newtheorem{theorem}{Definice}

\begin{document}
\begin{titlepage}

\begin{center}
    \Huge
	\textsc{Vysoké učení technické v~Brně}\\[0.5em]
    \huge
	{\textsc{Fakulta informačních technologií }}\\
	
	\vspace{\stretch{0.382}}
	\LARGE
	Typografie a publikování \,--\ 2. projekt\\[0.6em]
	Sazba dokumentů a matematických výrazů\\
	
	\vspace{\stretch{0.618}}
\end{center}

{\Large 2024 \hfill
Jaroslav Mervart (xmervaj00)}
\end{titlepage}
\newpage

\section*{Úvod}
V~této úloze si vysázíme titulní stranu a kousek matematického textu, v~němž se vyskytují například Definice~\ref{def:first} nebo rovnice~\eqref{eq:Rovnica_2} na straně~\pageref{eq:Rovnica_2}. Pro vytvořenítěchto odkazů používáme kombinace příkazů \verb|\label|, \verb|\ref|, \verb|\eqref| a \verb|\pageref|. Před odkazy patří nezlomitelná mezera. Pro zvýrazňování textu se používají příkazy \verb|\verb| a \verb|\emph|.

Titulní strana je vysázena prostředím \texttt{titlepage} a nadpis je v~optickém středu s~využitím \emph{přesného} zlatého řezu, který byl probrán na přednášce. Dále jsou na titulní straně čtyři různé velikosti písma a mezi dvojicemi řádků textu je použito řádkování se zadanou relativní velikostí 0,5\,em a 0,6\,em\footnote{Použijte správný typ mezery mezi číslem a jednotkou.}.

\section{Matematický text}
Matematické symboly a výrazy v~plynulém textu jsou v~prostředí math. Definice a věty sázíme v~prostředí definovaném příkazem \verb|\newtheorem| z~balíku amsthm. Tato prostředí obracejí význam \verb|\emph|: uvnitř textu sázeného kurzívou se zvýrazňuje písmem v~základním řezu. Někdy je vhodné použít konstrukci \verb|${}$| nebo \verb|\mbox{}|, která zabrání zalomení (matematického) textu. Pozor také na tvar i sklon řeckých písmen: srovnejte \verb|\epsilon| a \verb|\varepsilon|, \verb|\Xi| a \verb|\varXi|.

\begin{theorem}\label{def:first}
\emph{Konečný přepisovací stroj} neboli \emph{Mea\-ly\-ho automat} je definován jako uspořádaná pětice tvaru $M = (Q,\varSigma, \varGamma, \delta, q_0)$, kde:
\begin{itemize}
    \item Q je konečná množina \emph{stavů},
    \item $\varSigma$ je konečná \emph{vstupní abeceda}
    \item $\varGamma$ je konečná \emph{výstupní abeceda},
    \item $\delta:Q \times \varSigma \rightarrow Q \times \varGamma$ je totální přechodová funkce,
    \item $q_0 \in Q$ je \emph{počáteční stav}.
\end{itemize}
\end{theorem}
\subsection{Podsekce s~definicí}
Pomocí přechodové funkce $\delta$ zavedeme novou funkci $\delta^\ast$ pro překlad vstvstupních slov $u \in \varSigma^\ast$ do výstupních slov $w \in \varGamma^\ast$.

\begin{theorem}
Nechť $M=(Q,\varSigma,\varGamma,\delta,q_0)$ je Mealyho automat. \emph{Překládací funkce} $\delta^\ast:Q\times\varSigma^\ast \times \varGamma^\ast \rightarrow \varGamma^\ast$ je každý stav $q \in Q$, symbol $x \in \varSigma$, slova $u \in \varSigma^\ast$, $w \in \varGamma^\ast$ definována rekurentním předpisem:
\begin{itemize}
    \item $\delta^\ast(q,\varepsilon,w)= w$
    \item $\delta^\ast(q,xu,w)= \delta^\ast(q^\prime,u,wy), kde (q^\prime,y) = \delta(q,x)$
\end{itemize}
\end{theorem}

\subsection{Rovnice}
Složitější matematické formule sázíme mimo plynulý text pomocí prostředí \texttt{displaymath}. Lze umístit i více výrazů na jeden řádek, ale pak je třeba tyto vhodně oddělit, například pomocí \verb|\quad|, při dostatku místa i \verb|\qquad|.
\begin{displaymath}
    g^{a_n} \notin A^{B^n} \qquad y_0^1-\sqrt[5]{x+\sqrt[7]{y}} \qquad x > y^2 >= y^3 
\end{displaymath}

Velikost závorek a svislých čar je potřeba přizpůsobit jejich obsahu. Velikost lze stanovit explicitně, anebo pomocí \verb|\left| a \verb|\right|. Kombinační čísla sázejte makrem \verb|\binom|.
\begin{displaymath}
    \left|\bigcup P\right|=\sum\limits_{\emptyset\neq X \subseteq P} (-1)^{|X|-1}\left|\bigcap X\right|
\end{displaymath}
\begin{displaymath}
    F_{n+1}=\binom{n}{0}+\binom{n-1}{1}+\binom{n-2}{1}+\cdots+\binom{\lceil \frac{n}{2}\rceil}{\lfloor \frac{n}{2}\rfloor}
\end{displaymath}

V~rovnici~\eqref{eq:Rovnica_1} jsou tři typy závorek s~různou \emph{explicitně} definovanou velikostí. Obě rovnice mají svisle zarovnaná rovnítka. Použijte k~tomu vhodné prostředí.

\begin{eqnarray}
    \biggl( \Bigl\{ b\otimes \bigl[c_1 \oplus c_2 \bigr] \circ a \Bigr\} ^{\frac{2}{3}}\biggr) \;=\;\log_z x\label{eq:Rovnica_1}\\
    \int_a^b f(x)\, \mathrm{d}x \;=\; -\int_a^b f(y)\, \mathrm{d}y\label{eq:Rovnica_2}
\end{eqnarray}

V~této větě vidíme, jak se vysází proměnná určující limitu v~běžném textu: $\lim_{x\to\infty} f(m)$. Podobně je to i s~dalšími symboly jako $\bigcup_{N\in\mathcal{M}}N$ či $\sum_{i=1}^m x_i^2$. S~vynucením méně úsporné sazby příkazem \verb|\limits| budou vzorce vysázeny v~podobě $\lim\limits_{x\to\infty} f(m)$ a $\sum\limits_{i=1}^m x_i^2$.

\section{Matice}
Pro sázení matic se používá prostředí \texttt{array} a závorky s~výškou nastavenou pomocí \verb|\left|, \verb|\right|.
    $$ 
    D = \left|
    \begin{array}{cccc}
         a_{11} & a_{12} & \cdots & a_{1n} \\
         a_{21} & a_{22} & \cdots & a_{2n} \\
         \vdots & \vdots & \ddots & \vdots \\
         a_{m1} & a_{m2} & \cdots & a_{mn} \\
    \end{array}\right|
    = \left|
    \begin{array}{cc}
         x & y \\
         t & w \\
    \end{array}\right|
    = xw - yt
    $$
Prostředí \texttt{array} lze úspěšně využít i jinde, například na pravé straně následující rovnosti.
  $$\binom{n}{k} = \left\{
    \begin{array}{cl}
         \frac{n!}{k!(n-k)!} & \text{pro } 0\leq k\leq  n\\ 
         0 & \text{jinak}
    \end{array} \right.
   $$
\end{document}


